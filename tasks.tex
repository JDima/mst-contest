\documentclass[a4]{article}
\pagestyle{myheadings}

%%%%%%%%%%%%%%%%%%%
% Packages/Macros %
%%%%%%%%%%%%%%%%%%%
\usepackage{mathrsfs}


\usepackage{fancyhdr}
\pagestyle{fancy}
\lhead{}
\chead{}
\rhead{}
\lfoot{}
\cfoot{} 
\rfoot{\normalsize\thepage}
\renewcommand{\headrulewidth}{0pt}
\renewcommand{\footrulewidth}{0pt}
\newcommand{\RomanNumeralCaps}[1]
    {\MakeUppercase{\romannumeral #1}}

\usepackage{amssymb,latexsym}  % Standard packages
\usepackage[utf8]{inputenc}
\usepackage[russian]{babel}
\usepackage{MnSymbol}
\usepackage{mathrsfs}
\usepackage{amsmath,amsthm}
\usepackage{indentfirst}
\usepackage{graphicx}%,vmargin}
\usepackage{graphicx}
\graphicspath{{pictures/}} 
\usepackage{verbatim}
\usepackage{color}
\usepackage{color,colortbl}
\usepackage[nottoc,numbib]{tocbibind}
\usepackage{float}
\usepackage{multirow}
\usepackage{hhline}

\usepackage{listings}
\definecolor{codegreen}{rgb}{0,0.6,0}
\definecolor{codegray}{rgb}{1,1,1}
\definecolor{codepurple}{rgb}{0.58,0,0.82}
\definecolor{backcolour}{rgb}{0.95,0.95,0.92}
 
\lstdefinestyle{mystyle}{
    backgroundcolor=\color{backcolour},   
    commentstyle=\color{codegreen},
    keywordstyle=\color{magenta},
    numberstyle=\tiny\color{codegray},
    stringstyle=\color{codepurple},
    basicstyle=\footnotesize,
    breakatwhitespace=false,         
    breaklines=true,                 
    captionpos=b,                    
    keepspaces=true,                 
    numbers=left,                    
    numbersep=5pt,                  
    showspaces=false,                
    showstringspaces=false,
    showtabs=false,                  
    tabsize=2
}
 
\lstset{style=mystyle}

\usepackage{url}
\urldef\myurl\url{foo%.com}
\def\UrlBreaks{\do\/\do-}
\usepackage{breakurl}
\Urlmuskip=0mu plus 1mu



\DeclareGraphicsExtensions{.pdf,.png,.jpg}% -- настройка картинок

\usepackage{epigraph} %%% to make inspirational quotes.
\usepackage[all]{xy} %for XyPic'a
\usepackage{color} 
\usepackage{amscd} %для коммутативных диграмм
%\usepackage[colorlinks,urlcolor=red]{hyperref}

%\renewcommand{\baselinestretch}{1.5}
%\sloppy
%\usepackage{listings}
%\lstset{numbers=left}
%\setmarginsrb{2cm}{1.5cm}{1cm}{1.5cm}{0pt}{0mm}{0pt}{13mm}


\newtheorem{Lemma}{Лемма}[section]
\newtheorem{Proposition}{Предложение}[section]
\newtheorem{Theorem}{Теорема}[section]
\newtheorem{Corollary}{Следствие}[section]
\newtheorem{Remark}{Замечание}[section]
\newtheorem{Definition}{Определение}[section]
\newtheorem{Designations}{Обозначение}[section]




%%%%%%%%%%%%%%%%%%%%%%% 
%Подготовка оглавления% 
%%%%%%%%%%%%%%%%%%%%%%% 
\usepackage[titles]{tocloft}
\renewcommand{\cftdotsep}{2} %частота точек
\renewcommand\cftsecleader{\cftdotfill{\cftdotsep}}
\renewcommand{\cfttoctitlefont}{\hspace{0.38\textwidth} \LARGE\bfseries} 
\renewcommand{\cftsecaftersnum}{.}
\renewcommand{\cftsubsecaftersnum}{.}
\renewcommand{\cftbeforetoctitleskip}{-1em} 
\renewcommand{\cftaftertoctitle}{\mbox{}\hfill \\ \mbox{}\hfill{\footnotesize Стр.}\vspace{-0.5em}} 
%\renewcommand{\cftchapfont}{\normalsize\bfseries \MakeUppercase{\chaptername} } 
%\renewcommand{\cftsecfont}{\hspace{1pt}} 
\renewcommand{\cftsubsecfont}{\hspace{1pt}} 
%\renewcommand{\cftbeforechapskip}{1em} 
\renewcommand{\cftparskip}{3mm} %определяет величину отступа в оглавлении
\setcounter{tocdepth}{5} 
\renewcommand{\listoffigures}{\begingroup %добавляем номер в список иллюстраций
\tocsection
\tocfile{\listfigurename}{lof}
\endgroup}
\renewcommand{\listoftables}{\begingroup %добавляем номер в список иллюстраций
\tocsection
\tocfile{\listtablename}{lot}
\endgroup}


%\renewcommand{\thelikesection}{(\roman{likesection})}
%%%%%%%%%%%
% Margins %
%%%%%%%%%%%
\addtolength{\textwidth}{0.7in}
\textheight=630pt
\addtolength{\evensidemargin}{-0.4in}
\addtolength{\oddsidemargin}{-0.4in}
\addtolength{\topmargin}{-0.4in}

%%%%%%%%%%%%%%%%%%%%%%%%%%%%%%%%%%%
%%%%%%Переопределение chapter%%%%%% 
%%%%%%%%%%%%%%%%%%%%%%%%%%%%%%%%%%%
\newcommand{\empline}{\mbox{}\newline} 
\newcommand{\likechapterheading}[1]{ 
\begin{center} 
\textbf{\MakeUppercase{#1}} 
\end{center} 
\empline} 

%%%%%%%Запиливание переопределённого chapter в оглавление%%%%%% 
\makeatletter 
\renewcommand{\@dotsep}{2} 
\newcommand{\l@likechapter}[2]{{\bfseries\@dottedtocline{0}{0pt}{0pt}{#1}{#2}}} 
\makeatother 
\newcommand{\likechapter}[1]{ 
\likechapterheading{#1} 
\addcontentsline{toc}{likechapter}{\MakeUppercase{#1}}} 




\usepackage{xcolor}
\usepackage{hyperref}
\definecolor{linkcolor}{HTML}{000000} % цвет ссылок
\definecolor{urlcolor}{HTML}{AA1622} % цвет гиперссылок
 
\hypersetup{pdfstartview=FitH,  linkcolor=linkcolor,urlcolor=urlcolor, colorlinks=true}

%%%%%%%%%%%%
% Document %
%%%%%%%%%%%%

%%%%%%%%%%%%%%%%%%%%%%%%%%%%%
%%%%%%главы -- section*%%%%%%
%%%%section -- subsection%%%%
%subsection -- subsubsection%
%%%%%%%%%%%%%%%%%%%%%%%%%%%%%
\def \newstr {\medskip \par \noindent} 
\begin{document}



\section*{Задача 1}
\label{sec:orgb62fe60}
\subsection*{Постановка}
\label{sec:org37954e9}
В стране Турляндии есть \(N\) городов и \(M\) двусторонних дорог, каждая из которых соединяет ровно два города. Между двумя городами может быть более одной дороги. Дорога
может соединять город с самим собой, образуя петлю.
Все дороги находятся в плачевном состоянии. Для ремонта дороги необходимо \(g_i\) техники и \(s_i\) рабочей силы. Единица техники стоит \(G\), а единица рабочий силы - \(S\).
Необходимо найти минимальное количество денег необходимое для починок дорог так,
чтобы между каждой парой городов существовал хоты бы один безопасный путь.

\subsection*{Входные данные}
\label{sec:orgc51833b}
Первая строка входных данных содержит два целых числа \(N\) и \(M\) — количество городов
и количество дорог. Вторая строка содержит числа \(G\) и \(S\). Последующие \(M\) строк содержат
информацию о дорогах и то, что необходимо для её ремонта. Описание состоит из 4 чисел:
\(x_i\) - номер первого города, \(y_i\) - номер второго города, \(a_i\) - количество необходимой техники,
\(b_i\) - количество рабочей силы.

\subsection*{Выходные данные}
\label{sec:org91cd1c2}
Единственная строка должна содержать минимальное цену, которую нужно потратить
королю на покупку золотых и серебряных монет, чтобы достичь своей цели. Выведите -1,
в случае, если это невозможно.

\subsection*{Пример}
\label{sec:org2aeecb4}

\begin{table}[H]
\begin{center}
\begin{tabular}{|m{4cm}|m{4cm}|}
\hline
Входные данные & Выходные данные \\ \hline
3 3

2 1

1 2 10 15

1 2 4 20

1 3 5 1
&
30
\\ \hline
\end{tabular}
\end{center}
\end{table}

\pagebreak
\section*{Задача 2}
\label{sec:orgef181bd}
\subsection*{Постановка}
\label{sec:orgad8a20e}
Майк соединяет веснушки на спине отца пока он спит. Веснушки расположены в различных местах. Помогите Майку соединить точки так, чтобы минимизировать количество
используемых чернил. Майк соединяет точки, рисуя прямые линии между парами. Он
может поднимать ручку передвигая её по уже нарисованной линии, что не тратит чернила. Когда Майк закончил, должна быть последовательность соединенных линий от любой
веснушки до любой другой веснушки.
\subsection*{Входные данные}
\label{sec:orgc51833b}
В первой строке записано \(n\) - количество веснушек. За каждой веснушкой следует линия. Каждая следующая строка содержит два действительных числа, указывающих координаты \((x, y)\) веснушки.

\subsection*{Выходные данные}
\label{sec:orgf9da829}
Необходимо вывести минимальную общую длину линий чернил с двумя знаками после запятой.
\subsection*{Пример}
\label{sec:orgd7d348d}

\begin{table}[H]
\begin{center}
\begin{tabular}{|m{4cm}|m{4cm}|}
\hline
Входные данные & Выходные данные \\ \hline
3

1.0 1.0

2.0 2.0

2.0 4.0
&
3.41
\\ \hline
\end{tabular}
\end{center}
\end{table}

\pagebreak
\section*{Задача 3}
\label{sec:org570b899}
\subsection*{Постановка}
\label{sec:orga2b5149}
Рассмотрим задачу выбора набора \(T\) высокоскоростных линий для соединения \(N\) компьютерных узлов из совокупности \(M\) высокоскоростных линий, каждая из которых соединяет пару компьютерных узлов.
Каждая высокоскоростная линия имеет заданную ежемесячную стоимость, и необходимо свести к минимуму общую стоимость подключения \(N\) компьютерных сайтов, где общая
стоимость - это сумма стоимости каждой линии, включенной в набор \(T\). Рассмотрим далее,
что эта проблема ранее было решено для набора из \(N\) компьютерных узлов и \(M\) высокоскоростных линий, но недавно стало доступно несколько \(K\) новых высокоскоростных линий.
Ваша цель - вычислить новый набор \(T\), который может дать более низкую стоимость,
чем исходный набор \(T\), из-за дополнительных K новых высокоскоростных линий и когда
доступно \(M + K\) высокоскоростных линий
\label{sec:orged795e8}

\subsection*{Входные данные}
\label{sec:orgeb4908d}
Данные подаются следующим образом:\
\begin{itemize}
    \item Количество компьютерных сайтов \(N\).
    \item Максимумом по числам, стоящим на четных позициях.
    \item Множество \(T\) предварительно выбранных высокоскоростных линий, состоящих из \(N - 1\) линий, каждая из которых описывает высокоскоростную линию и содержит номера двух компьютерных сайтов, которые линия соединяет, и ежемесячная стоимость использования этой линии.
    \item Строка, содержащая количество \(K\) новых дополнительных линий.
    \item \(K\) строк, каждая из которых описывает новую высокоскоростную линию и содержит номера двух компьютеров. к которым линия подключается, и ежемесячная стоимость использования этой линии.
    \item Строка, содержащая количество \(M\) изначально доступных высокоскоростных линий.
    \item \(M\) линий, каждая из которых описывает одну из изначально доступных высокоскоростных линий и содержит номера двух компьютерных сайтов, к которым линия подключает, а также ежемесячную стоимость использования этой линии.
\end{itemize}
\subsection*{Выходные данные}
\label{sec:orged795e8}
Выход должен содержать одну строку, содержащую исходную стоимость соединения
\(N\) компьютерных узлов с \(M\) высокоскоростными линиями, а другую строку - новую стоимость соединения \(N\) компьютерных узлов с \(M + K\) высокоскоростными линиями. Необходимо вывести минимальную общую длину чернил с двумя знаками после запятой.
\subsection*{Пример}
\label{sec:org6a26c04}

\begin{table}[H]
\begin{center}
\begin{tabular}{|m{4cm}|m{4cm}|}
\hline
Входные данные & Выходные данные \\ \hline
5

1 2 5

1 3 5

1 4 5

1 5 5

1

2 3 2

6

1 2 5

1 3 5

1 4 5

1 5 5

3 4 8

4 5 
&
20

17
\\ \hline
\end{tabular}
\end{center}
\end{table}

\pagebreak
\section*{Задача 4}
\label{sec:orgb1f46a6}
\subsection*{Постановка}
\label{sec:orge854c50}
После встречи с инвесторами, стартап по созданию платформы для умного сжатия данных получил контракт на первый этап финансирования. Для увеличения эффективности выполнеиня работы инвесторы настояли на изменении организационной структуры стартапа. На момент подписания договора в стартапе работало \(n\) человек. Теперь в стартапе необходимо построить иерархию отношений. То есть у каждого работника, кроме исполнительного директора, должен быть ровно один тимлид.
Есть \(m\) заявлений о том, что работник \(a_i\) готов быть начальником работника \(b_i\) за доплату
\(c_i\) Для каждого заявления известно, что квалификация подающего заявление больше, чем
квалификация того человека, чьим начальником он готов стать.
Помогите определить минимальную стоимость создания иерархии, либо выяснить, что
это невозможно
\subsection*{Входные данные}
\label{sec:orge854c50}
В первой строке содержится целое число \(n\) — число сотрудников стартапа. В следующей строке содержится n чисел \(q_j\) — значения квалификации сотрудников. Далее идет
количество поданных заявлений \(m\). Следующие \(m\) строк содержат заявления, каждое из
которых задается тремя числами: \(a_i\)
, \(b_i\) и \(c_i\).

\subsection*{Выходные данные}
\label{sec:org1ab7414}
Выведите минимальную стоимость создания иерархии или -1, если создать иерархию
невозможно.

\subsection*{Пример 1}
\label{sec:org25482f8}

\begin{table}[H]
\begin{center}
\begin{tabular}{|m{4cm}|m{4cm}|}
\hline
Входные данные & Выходные данные \\ \hline
4

7 2 3 1

4

1 2 5

2 4 1

3 4 1

1 3 5
&
11
\\ \hline
\end{tabular}
\end{center}
\end{table}

\begin{table}[H]
\begin{center}
\begin{tabular}{|m{4cm}|m{4cm}|}
\hline
Входные данные & Выходные данные \\ \hline
4

7 2 3 1

4

1 2 5

2 4 1

3 4 1

1 3 5
&
-1
\\ \hline
\end{tabular}
\end{center}
\end{table}
\pagebreak
\end{document}
